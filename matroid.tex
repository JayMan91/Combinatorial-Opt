\documentclass[options]{article}
\usepackage{mathtools}
\usepackage{amssymb}
\usepackage{amsmath}
\usepackage{enumitem}
\usepackage{amsthm}
\setlist[itemize]{nosep}
\setlist[enumerate]{nosep}
\newtheorem{theorem}{Theorem}
\theoremstyle{definition}
\newtheorem{definition}{Definition}
\newcommand{\independentI}{\mathcal{I}}
\DeclareMathOperator*{\convhull}{convhull} 
\begin{document}
	\title{Discrete Optimization}
	\maketitle
\section{Matroid}
\begin{definition}
 A \emph{Matroid} $M$ is defined by a set-system pair ($S$,
$\mathcal{I}$), where $S$ is a ground set and $\mathcal{I}$ is family of subsets of $S$.
For $M =$ ($S$,$\mathcal{I}$) to be a Matroid the the elements of $\mathcal{I}$
should obey the following:
\begin{enumerate}
\item $\phi$ $\in \mathcal{I}$
\item If $A \in \mathcal{I}$ and $B \subseteq A$ $\implies B \in \mathcal{I}$
\item If $A, B \in \mathcal{I}$ and $|B| > |A|$ then $ \exists \ b \in B \backslash  A$   such that $A \cup \{ b \} \in \mathcal{I}$ 
\end{enumerate}
\end{definition}

\begin{itemize}

\item If $ T \subseteq S$ is not an element of $\mathcal{I}$ $T$ is called a dependent set.
\item An element $T \in \mathcal{I}$ is a basis if it is a maximal independent set.
\item A circuit is a minimal independent set.
\item A set $T \in \mathcal{I}$  is a spanning set if T contains a basis. 
\end{itemize}
\paragraph{example} $S$ = $\{ a,b,c \}$ and $\independentI =\{ \{ a,b \},\{ a \}, \{ b \},\phi \}$ then $\{c \},\{a,c\},\{b,c\},\{a,b,c\}$ are dependent set  $\{c\}$ is a circuit and $\{ a,b\}$ is the a basis.
\paragraph{Rank function} The rank function $r_M : 2^S \rightarrow \mathbb{Z} $ of matroid $M$ is defined as $r_M (T) = max \{ |P|: P \subseteq T,P \in \independentI \}$

With the help of rank function, we can define span as
\begin{definition}
For a set $T\ \subseteq S \ span(T) = \{ e \in S: rank(T) = rank(T \cup {e}) \} $
\end{definition}

\begin{definition}
$T \subseteq S$ is said to be \emph{closed} if $span(T) = T$.
\end{definition}
\begin{theorem}
The rank function of any matroid is submodular.
\end{theorem}

\begin{proof}
    We have to show
    \[
    r_M (X)+r_M (Y) \geq r_M (X \cup Y) + r_M (X \cap Y), \forall X,Y \in 2^S
    \]
    Let $J$ be a maximal independent subset of $ X \cap Y$. This $J$ can be extended to $J_X$ - maximal independent subset of $X$ and $J_{XY}$ -  maximal independent subset of $X \cup Y$ 
    
  So, $r_M(X) = |J_X|$, $|r_M(X \cup Y)| = |J_{XY}|$, $r_M(X \cap Y)= |J|$

We have to show $r_M(Y) \geq |J_{XY}|+|J|-|J_X|$

$|J_{XY}|+|J|-|J_X| = |J_{XY}\backslash (J \backslash J_X)| $

By construction,
$ J \backslash J_X = J \backslash Y$.

Hence $J_{XY}\backslash (J \backslash J_X) = J_{XY}\backslash (J \backslash Y) = J_{XY} \cap Y$

Now, 
\begin{align*}
r_M (Y) \geq |J_{XY} \cap Y| &= |J_{XY}\backslash (J \backslash J_X)| \\
 &=  |J_{XY}|+|J|-|J_X|\\
  &= r_M (X \cup Y) + r_M (X \cap Y)- r_M(X)  
\end{align*}
\end{proof}
\begin{definition}
The indicator or incidence vector of $ T \subseteq S$ is $\chi (T) := (x_s)_{s \in S}$ such that $x_s =1$  if $s \in T$ and $x_s =0$  if $s \notin T$   
\end{definition}

\begin{definition}
For a matroid $M = (S,\independentI)$, the \emph{matroid polytope} is defined as the convex hull of all the indicator(incidence) vectors of the independent sets of $M$. $P(M) = \convhull \{ \chi (T): T \in  \independentI \} $
\end{definition}
\begin{theorem}
For a vector $x \in \mathcal{R}^S$ let $x(T)$ denotes $\sum_{i \in T} x_i$, then 

$P(M) = \{ x \in \mathcal{R}_{+}^{S} : x(T) \leq r_M(T) \forall T \subseteq S \}$
\end{theorem}
\section{Extensions}
An extension of a set function $f: 2^S \rightarrow \mathcal{R}$ is some functions functions $[0,1]^S \rightarrow \mathcal{R}$ that agress with $f$ on the vertices of the hypercube.
\subsection{Lov{\'a}sz Extension}
Given a set function $f: 2^S \rightarrow \mathcal{R}$, for a point $ x \in [0,1]^S$, first order the componnets in decreasing order $x_{j_1} \geq ...\geq x_{j_p}$ where $(j_1,...,j_p)$ is a permutation. Then the \textbf{Lov{\'a}sz Extension} is defined as :
\[
\mathcal{L}_f (x) = \sum_{k=1}^n x_{j_k} [S({j_1,...,j_k}) - S({j_1,...,j_{k-1}})]
\]
\paragraph{example} 
For $p=2$,
\[
\mathcal{L}_f (x)=\begin{cases}
   f(\{1\})x_1+ x_2(f(\{1,2\})- f(\{1\}),& \text{if } x_1 \geq x_2\\
     f(\{2\})x_2+ x_1(f(\{1,2\})- f(\{2\}),& \text{if } x_2 \geq x_1
\end{cases}
\] 
\end{document}
\documentclass[options]{article}
\usepackage{mathtools}
\usepackage{amssymb}
\usepackage{amsmath}
\usepackage{enumitem}
\usepackage{amsthm}
\setlist[itemize]{nosep}
\setlist[enumerate]{nosep}
\newtheorem{theorem}{Theorem}
\newcommand{\independentI}{\mathcal{I}} 
\begin{document}
	\title{Discrete Optimization}
	\maketitle
\section{Matroid}
\paragraph{Matroid} A \emph{Matroid} $M$ is defined by a set-system pair ($S$,
$\mathcal{I}$), where $S$ is a ground set and $\mathcal{I}$ is family of subsets of $S$.

For $M =$ ($S$,$\mathcal{I}$) to be a Matroid the the elements of $\mathcal{I}$
should obey the following:
\begin{enumerate}
\item $\phi$ $\in \mathcal{I}$
\item If $A \in \mathcal{I}$ and $B \subseteq A$ $\implies B \in \mathcal{I}$
\item If $A, B \in \mathcal{I}$ and $|B| > |A|$ then $ \exists \ b \in B \backslash  A$   such that $A \cup \{ b \} \in \mathcal{I}$ 
\end{enumerate}
\begin{itemize}
\item If $ T \subseteq S$ is not an element of $\mathcal{I}$ $T$ is called a dependent set.
\item An element $T \in \mathcal{I}$ is a basis if it is a maximal independent set.
\item A circuit is a minimal independent set.
\item A set $T \in \mathcal{I}$  is a spanning set if T contains a basis. 
\end{itemize}
\paragraph{example} $S$ = $\{ a,b,c \}$ and $\independentI =\{ \{ a,b \},\{ a \}, \{ b \},\phi \}$ then $\{c \},\{a,c\},\{b,c\},\{a,b,c\}$ are dependent set  $\{c\}$ is a circuit and $\{ a,b\}$ is the a basis.
\paragraph{Rank function} The rank function $r_M : 2^S \rightarrow \mathbb{Z} $ of matroid $M$ is defined as $r_M (T) = max \{ |P|: P \subseteq T,P \in \independentI \}$

\begin{theorem}
The rank function of any matroid is submodular.
\end{theorem}

\begin{proof}
    We have to show
    \[
    r_M (X)+r_M (Y) \geq r_M (X \cup Y) + r_M (X \cap Y), \forall X,Y \in 2^S
    \]
    Let $J$ be a maximal independent subset of $ X \cap Y$. This $J$ can be extended to $J_X$ - maximal independent subset of $X$ and $J_{XY}$ -  maximal independent subset of $X \cup Y$ 
    
  So, $r_M(X) = |J_X|$, $|r_M(X \cup Y)| = |J_{XY}|$, $r_M(X \cap Y)= |J|$

We have to show $r_M(Y) \geq |J_{XY}|+|J|-|J_X|$

$|J_{XY}|+|J|-|J_X| = |J_{XY}\backslash (J \backslash J_X)| $

By construction,
$ J \backslash J_X = J \backslash Y$.

Hence $J_{XY}\backslash (J \backslash J_X) = J_{XY}\backslash (J \backslash Y) = J_{XY} \cap Y$

Now, 
\begin{align*}
r_M (Y) \geq |J_{XY} \cap Y| &= |J_{XY}\backslash (J \backslash J_X)| \\
 &=  |J_{XY}|+|J|-|J_X|\\
  &= r_M (X \cup Y) + r_M (X \cap Y)- r_M(X)  
\end{align*}
\end{proof}

\end{document}